\documentclass[]{article}
\usepackage{lmodern}
\usepackage{amssymb,amsmath}
\usepackage{ifxetex,ifluatex}
\usepackage{fixltx2e} % provides \textsubscript
\ifnum 0\ifxetex 1\fi\ifluatex 1\fi=0 % if pdftex
  \usepackage[T1]{fontenc}
  \usepackage[utf8]{inputenc}
\else % if luatex or xelatex
  \ifxetex
    \usepackage{mathspec}
  \else
    \usepackage{fontspec}
  \fi
  \defaultfontfeatures{Ligatures=TeX,Scale=MatchLowercase}
\fi
% use upquote if available, for straight quotes in verbatim environments
\IfFileExists{upquote.sty}{\usepackage{upquote}}{}
% use microtype if available
\IfFileExists{microtype.sty}{%
\usepackage{microtype}
\UseMicrotypeSet[protrusion]{basicmath} % disable protrusion for tt fonts
}{}
\usepackage[margin=2.54cm]{geometry}
\usepackage{hyperref}
\hypersetup{unicode=true,
            pdftitle={Assignment 6: Generalized Linear Models},
            pdfauthor={Lindsay Roth},
            pdfborder={0 0 0},
            breaklinks=true}
\urlstyle{same}  % don't use monospace font for urls
\usepackage{color}
\usepackage{fancyvrb}
\newcommand{\VerbBar}{|}
\newcommand{\VERB}{\Verb[commandchars=\\\{\}]}
\DefineVerbatimEnvironment{Highlighting}{Verbatim}{commandchars=\\\{\}}
% Add ',fontsize=\small' for more characters per line
\usepackage{framed}
\definecolor{shadecolor}{RGB}{248,248,248}
\newenvironment{Shaded}{\begin{snugshade}}{\end{snugshade}}
\newcommand{\KeywordTok}[1]{\textcolor[rgb]{0.13,0.29,0.53}{\textbf{#1}}}
\newcommand{\DataTypeTok}[1]{\textcolor[rgb]{0.13,0.29,0.53}{#1}}
\newcommand{\DecValTok}[1]{\textcolor[rgb]{0.00,0.00,0.81}{#1}}
\newcommand{\BaseNTok}[1]{\textcolor[rgb]{0.00,0.00,0.81}{#1}}
\newcommand{\FloatTok}[1]{\textcolor[rgb]{0.00,0.00,0.81}{#1}}
\newcommand{\ConstantTok}[1]{\textcolor[rgb]{0.00,0.00,0.00}{#1}}
\newcommand{\CharTok}[1]{\textcolor[rgb]{0.31,0.60,0.02}{#1}}
\newcommand{\SpecialCharTok}[1]{\textcolor[rgb]{0.00,0.00,0.00}{#1}}
\newcommand{\StringTok}[1]{\textcolor[rgb]{0.31,0.60,0.02}{#1}}
\newcommand{\VerbatimStringTok}[1]{\textcolor[rgb]{0.31,0.60,0.02}{#1}}
\newcommand{\SpecialStringTok}[1]{\textcolor[rgb]{0.31,0.60,0.02}{#1}}
\newcommand{\ImportTok}[1]{#1}
\newcommand{\CommentTok}[1]{\textcolor[rgb]{0.56,0.35,0.01}{\textit{#1}}}
\newcommand{\DocumentationTok}[1]{\textcolor[rgb]{0.56,0.35,0.01}{\textbf{\textit{#1}}}}
\newcommand{\AnnotationTok}[1]{\textcolor[rgb]{0.56,0.35,0.01}{\textbf{\textit{#1}}}}
\newcommand{\CommentVarTok}[1]{\textcolor[rgb]{0.56,0.35,0.01}{\textbf{\textit{#1}}}}
\newcommand{\OtherTok}[1]{\textcolor[rgb]{0.56,0.35,0.01}{#1}}
\newcommand{\FunctionTok}[1]{\textcolor[rgb]{0.00,0.00,0.00}{#1}}
\newcommand{\VariableTok}[1]{\textcolor[rgb]{0.00,0.00,0.00}{#1}}
\newcommand{\ControlFlowTok}[1]{\textcolor[rgb]{0.13,0.29,0.53}{\textbf{#1}}}
\newcommand{\OperatorTok}[1]{\textcolor[rgb]{0.81,0.36,0.00}{\textbf{#1}}}
\newcommand{\BuiltInTok}[1]{#1}
\newcommand{\ExtensionTok}[1]{#1}
\newcommand{\PreprocessorTok}[1]{\textcolor[rgb]{0.56,0.35,0.01}{\textit{#1}}}
\newcommand{\AttributeTok}[1]{\textcolor[rgb]{0.77,0.63,0.00}{#1}}
\newcommand{\RegionMarkerTok}[1]{#1}
\newcommand{\InformationTok}[1]{\textcolor[rgb]{0.56,0.35,0.01}{\textbf{\textit{#1}}}}
\newcommand{\WarningTok}[1]{\textcolor[rgb]{0.56,0.35,0.01}{\textbf{\textit{#1}}}}
\newcommand{\AlertTok}[1]{\textcolor[rgb]{0.94,0.16,0.16}{#1}}
\newcommand{\ErrorTok}[1]{\textcolor[rgb]{0.64,0.00,0.00}{\textbf{#1}}}
\newcommand{\NormalTok}[1]{#1}
\usepackage{graphicx,grffile}
\makeatletter
\def\maxwidth{\ifdim\Gin@nat@width>\linewidth\linewidth\else\Gin@nat@width\fi}
\def\maxheight{\ifdim\Gin@nat@height>\textheight\textheight\else\Gin@nat@height\fi}
\makeatother
% Scale images if necessary, so that they will not overflow the page
% margins by default, and it is still possible to overwrite the defaults
% using explicit options in \includegraphics[width, height, ...]{}
\setkeys{Gin}{width=\maxwidth,height=\maxheight,keepaspectratio}
\IfFileExists{parskip.sty}{%
\usepackage{parskip}
}{% else
\setlength{\parindent}{0pt}
\setlength{\parskip}{6pt plus 2pt minus 1pt}
}
\setlength{\emergencystretch}{3em}  % prevent overfull lines
\providecommand{\tightlist}{%
  \setlength{\itemsep}{0pt}\setlength{\parskip}{0pt}}
\setcounter{secnumdepth}{0}
% Redefines (sub)paragraphs to behave more like sections
\ifx\paragraph\undefined\else
\let\oldparagraph\paragraph
\renewcommand{\paragraph}[1]{\oldparagraph{#1}\mbox{}}
\fi
\ifx\subparagraph\undefined\else
\let\oldsubparagraph\subparagraph
\renewcommand{\subparagraph}[1]{\oldsubparagraph{#1}\mbox{}}
\fi

%%% Use protect on footnotes to avoid problems with footnotes in titles
\let\rmarkdownfootnote\footnote%
\def\footnote{\protect\rmarkdownfootnote}

%%% Change title format to be more compact
\usepackage{titling}

% Create subtitle command for use in maketitle
\newcommand{\subtitle}[1]{
  \posttitle{
    \begin{center}\large#1\end{center}
    }
}

\setlength{\droptitle}{-2em}

  \title{Assignment 6: Generalized Linear Models}
    \pretitle{\vspace{\droptitle}\centering\huge}
  \posttitle{\par}
    \author{Lindsay Roth}
    \preauthor{\centering\large\emph}
  \postauthor{\par}
    \date{}
    \predate{}\postdate{}
  

\begin{document}
\maketitle

\subsection{OVERVIEW}\label{overview}

This exercise accompanies the lessons in Environmental Data Analytics
(ENV872L) on generalized linear models.

\subsection{Directions}\label{directions}

\begin{enumerate}
\def\labelenumi{\arabic{enumi}.}
\tightlist
\item
  Change ``Student Name'' on line 3 (above) with your name.
\item
  Use the lesson as a guide. It contains code that can be modified to
  complete the assignment.
\item
  Work through the steps, \textbf{creating code and output} that fulfill
  each instruction.
\item
  Be sure to \textbf{answer the questions} in this assignment document.
  Space for your answers is provided in this document and is indicated
  by the ``\textgreater{}'' character. If you need a second paragraph be
  sure to start the first line with ``\textgreater{}''. You should
  notice that the answer is highlighted in green by RStudio.
\item
  When you have completed the assignment, \textbf{Knit} the text and
  code into a single PDF file. You will need to have the correct
  software installed to do this (see Software Installation Guide) Press
  the \texttt{Knit} button in the RStudio scripting panel. This will
  save the PDF output in your Assignments folder.
\item
  After Knitting, please submit the completed exercise (PDF file) to the
  dropbox in Sakai. Please add your last name into the file name (e.g.,
  ``Salk\_A06\_GLMs.pdf'') prior to submission.
\end{enumerate}

The completed exercise is due on Tuesday, 26 February, 2019 before class
begins.

\subsection{Set up your session}\label{set-up-your-session}

\begin{enumerate}
\def\labelenumi{\arabic{enumi}.}
\item
  Set up your session. Upload the EPA Ecotox dataset for Neonicotinoids
  and the NTL-LTER raw data file for chemistry/physics.
\item
  Build a ggplot theme and set it as your default theme.
\end{enumerate}

\begin{Shaded}
\begin{Highlighting}[]
\CommentTok{#1}
\KeywordTok{getwd}\NormalTok{()}
\end{Highlighting}
\end{Shaded}

\begin{verbatim}
## [1] "/Users/lindsayroth/Documents/MEM 1st Year/Spring 2019/Env_Data_Analytics/Env_Data_Analytics"
\end{verbatim}

\begin{Shaded}
\begin{Highlighting}[]
\KeywordTok{library}\NormalTok{(tidyverse)}
\end{Highlighting}
\end{Shaded}

\begin{verbatim}
## -- Attaching packages --------------------------------------------------------------------------- tidyverse 1.2.1 --
\end{verbatim}

\begin{verbatim}
## v ggplot2 3.1.0     v purrr   0.3.0
## v tibble  2.0.1     v dplyr   0.7.8
## v tidyr   0.8.2     v stringr 1.3.1
## v readr   1.3.1     v forcats 0.3.0
\end{verbatim}

\begin{verbatim}
## -- Conflicts ------------------------------------------------------------------------------ tidyverse_conflicts() --
## x dplyr::filter() masks stats::filter()
## x dplyr::lag()    masks stats::lag()
\end{verbatim}

\begin{Shaded}
\begin{Highlighting}[]
\KeywordTok{library}\NormalTok{(viridis)}
\end{Highlighting}
\end{Shaded}

\begin{verbatim}
## Loading required package: viridisLite
\end{verbatim}

\begin{Shaded}
\begin{Highlighting}[]
\KeywordTok{library}\NormalTok{(RColorBrewer)}
\KeywordTok{library}\NormalTok{(colormap)}

\NormalTok{Ecotox <-}\StringTok{ }\KeywordTok{read.csv}\NormalTok{(}\StringTok{"./Data/Raw/ECOTOX_Neonicotinoids_Mortality_raw.csv"}\NormalTok{)}
\NormalTok{NTL.chem.physics <-}\StringTok{ }\KeywordTok{read.csv}\NormalTok{(}\StringTok{"./Data/Raw/NTL-LTER_Lake_ChemistryPhysics_Raw.csv"}\NormalTok{)}

\CommentTok{#2}
\NormalTok{lindsay_theme <-}\StringTok{ }\KeywordTok{theme_classic}\NormalTok{(}\DataTypeTok{base_size =} \DecValTok{12}\NormalTok{) }\OperatorTok{+}
\StringTok{  }\KeywordTok{theme}\NormalTok{(}\DataTypeTok{axis.text =} \KeywordTok{element_text}\NormalTok{(}\DataTypeTok{color =} \StringTok{"black"}\NormalTok{), }
        \DataTypeTok{legend.position =} \StringTok{"right"}\NormalTok{) }
\KeywordTok{theme_set}\NormalTok{(lindsay_theme)}
\end{Highlighting}
\end{Shaded}

\subsection{Neonicotinoids test}\label{neonicotinoids-test}

Research question: Were studies on various neonicotinoid chemicals
conducted in different years?

\begin{enumerate}
\def\labelenumi{\arabic{enumi}.}
\setcounter{enumi}{2}
\item
  Generate a line of code to determine how many different chemicals are
  listed in the Chemical.Name column.
\item
  Are the publication years associated with each chemical
  well-approximated by a normal distribution? Run the appropriate test
  and also generate a frequency polygon to illustrate the distribution
  of counts for each year, divided by chemical name. Bonus points if you
  can generate the results of your test from a pipe function. No need to
  make this graph pretty.
\item
  Is there equal variance among the publication years for each chemical?
  Hint: var.test is not the correct function.
\end{enumerate}

\begin{Shaded}
\begin{Highlighting}[]
\CommentTok{#3}
\KeywordTok{unique}\NormalTok{(Ecotox}\OperatorTok{$}\NormalTok{Chemical.Name)}
\end{Highlighting}
\end{Shaded}

\begin{verbatim}
## [1] Imidacloprid Thiacloprid  Thiamethoxam Acetamiprid  Clothianidin
## [6] Dinotefuran  Nitenpyram   Nithiazine   Imidaclothiz
## 9 Levels: Acetamiprid Clothianidin Dinotefuran ... Thiamethoxam
\end{verbatim}

\begin{Shaded}
\begin{Highlighting}[]
\CommentTok{#There are 9 different chemicals in the Chemical.Name column.}

\CommentTok{#4}
\KeywordTok{shapiro.test}\NormalTok{(Ecotox}\OperatorTok{$}\NormalTok{Pub..Year[Ecotox}\OperatorTok{$}\NormalTok{Chemical.Name }\OperatorTok{==}\StringTok{ "Imidacloprid"}\NormalTok{])}
\end{Highlighting}
\end{Shaded}

\begin{verbatim}
## 
##  Shapiro-Wilk normality test
## 
## data:  Ecotox$Pub..Year[Ecotox$Chemical.Name == "Imidacloprid"]
## W = 0.88178, p-value < 2.2e-16
\end{verbatim}

\begin{Shaded}
\begin{Highlighting}[]
\KeywordTok{shapiro.test}\NormalTok{(Ecotox}\OperatorTok{$}\NormalTok{Pub..Year[Ecotox}\OperatorTok{$}\NormalTok{Chemical.Name }\OperatorTok{==}\StringTok{ "Thiacloprid"}\NormalTok{])}
\end{Highlighting}
\end{Shaded}

\begin{verbatim}
## 
##  Shapiro-Wilk normality test
## 
## data:  Ecotox$Pub..Year[Ecotox$Chemical.Name == "Thiacloprid"]
## W = 0.7669, p-value = 1.118e-11
\end{verbatim}

\begin{Shaded}
\begin{Highlighting}[]
\KeywordTok{shapiro.test}\NormalTok{(Ecotox}\OperatorTok{$}\NormalTok{Pub..Year[Ecotox}\OperatorTok{$}\NormalTok{Chemical.Name }\OperatorTok{==}\StringTok{ "Thiamethoxam"}\NormalTok{])}
\end{Highlighting}
\end{Shaded}

\begin{verbatim}
## 
##  Shapiro-Wilk normality test
## 
## data:  Ecotox$Pub..Year[Ecotox$Chemical.Name == "Thiamethoxam"]
## W = 0.7071, p-value < 2.2e-16
\end{verbatim}

\begin{Shaded}
\begin{Highlighting}[]
\KeywordTok{shapiro.test}\NormalTok{(Ecotox}\OperatorTok{$}\NormalTok{Pub..Year[Ecotox}\OperatorTok{$}\NormalTok{Chemical.Name }\OperatorTok{==}\StringTok{ "Acetamiprid"}\NormalTok{])}
\end{Highlighting}
\end{Shaded}

\begin{verbatim}
## 
##  Shapiro-Wilk normality test
## 
## data:  Ecotox$Pub..Year[Ecotox$Chemical.Name == "Acetamiprid"]
## W = 0.90191, p-value = 5.706e-08
\end{verbatim}

\begin{Shaded}
\begin{Highlighting}[]
\KeywordTok{shapiro.test}\NormalTok{(Ecotox}\OperatorTok{$}\NormalTok{Pub..Year[Ecotox}\OperatorTok{$}\NormalTok{Chemical.Name }\OperatorTok{==}\StringTok{ "Clothianidin"}\NormalTok{])}
\end{Highlighting}
\end{Shaded}

\begin{verbatim}
## 
##  Shapiro-Wilk normality test
## 
## data:  Ecotox$Pub..Year[Ecotox$Chemical.Name == "Clothianidin"]
## W = 0.69577, p-value = 4.287e-11
\end{verbatim}

\begin{Shaded}
\begin{Highlighting}[]
\KeywordTok{shapiro.test}\NormalTok{(Ecotox}\OperatorTok{$}\NormalTok{Pub..Year[Ecotox}\OperatorTok{$}\NormalTok{Chemical.Name }\OperatorTok{==}\StringTok{ "Dinotefuran"}\NormalTok{])}
\end{Highlighting}
\end{Shaded}

\begin{verbatim}
## 
##  Shapiro-Wilk normality test
## 
## data:  Ecotox$Pub..Year[Ecotox$Chemical.Name == "Dinotefuran"]
## W = 0.82848, p-value = 8.83e-07
\end{verbatim}

\begin{Shaded}
\begin{Highlighting}[]
\KeywordTok{shapiro.test}\NormalTok{(Ecotox}\OperatorTok{$}\NormalTok{Pub..Year[Ecotox}\OperatorTok{$}\NormalTok{Chemical.Name }\OperatorTok{==}\StringTok{ "Nitenpyram"}\NormalTok{])}
\end{Highlighting}
\end{Shaded}

\begin{verbatim}
## 
##  Shapiro-Wilk normality test
## 
## data:  Ecotox$Pub..Year[Ecotox$Chemical.Name == "Nitenpyram"]
## W = 0.79592, p-value = 0.0005686
\end{verbatim}

\begin{Shaded}
\begin{Highlighting}[]
\KeywordTok{shapiro.test}\NormalTok{(Ecotox}\OperatorTok{$}\NormalTok{Pub..Year[Ecotox}\OperatorTok{$}\NormalTok{Chemical.Name }\OperatorTok{==}\StringTok{ "Nithiazine"}\NormalTok{])}
\end{Highlighting}
\end{Shaded}

\begin{verbatim}
## 
##  Shapiro-Wilk normality test
## 
## data:  Ecotox$Pub..Year[Ecotox$Chemical.Name == "Nithiazine"]
## W = 0.75938, p-value = 0.0001235
\end{verbatim}

\begin{Shaded}
\begin{Highlighting}[]
\KeywordTok{shapiro.test}\NormalTok{(Ecotox}\OperatorTok{$}\NormalTok{Pub..Year[Ecotox}\OperatorTok{$}\NormalTok{Chemical.Name }\OperatorTok{==}\StringTok{ "Imidaclothiz"}\NormalTok{])}
\end{Highlighting}
\end{Shaded}

\begin{verbatim}
## 
##  Shapiro-Wilk normality test
## 
## data:  Ecotox$Pub..Year[Ecotox$Chemical.Name == "Imidaclothiz"]
## W = 0.68429, p-value = 0.00093
\end{verbatim}

\begin{Shaded}
\begin{Highlighting}[]
\CommentTok{#The publication years associated with each chemical are not well approximated }
\CommentTok{#by the normal distribution because all of the p-values resulting from the}
\CommentTok{#Shapiro-Wilkes test are less than 0.001}

\KeywordTok{qqnorm}\NormalTok{(Ecotox}\OperatorTok{$}\NormalTok{Pub..Year); }\KeywordTok{qqline}\NormalTok{(Ecotox}\OperatorTok{$}\NormalTok{Pub..Year)}
\end{Highlighting}
\end{Shaded}

\includegraphics{LRoth_A06_GLMs_files/figure-latex/unnamed-chunk-2-1.pdf}

\begin{Shaded}
\begin{Highlighting}[]
\NormalTok{Ecotox.Freq <-}\StringTok{ }\KeywordTok{ggplot}\NormalTok{(Ecotox, }\KeywordTok{aes}\NormalTok{(}\DataTypeTok{x =}\NormalTok{ Pub..Year, }\DataTypeTok{color =}\NormalTok{ Chemical.Name)) }\OperatorTok{+}
\StringTok{  }\KeywordTok{geom_freqpoly}\NormalTok{(}\DataTypeTok{stat =} \StringTok{"count"}\NormalTok{)}
\KeywordTok{print}\NormalTok{(Ecotox.Freq)}
\end{Highlighting}
\end{Shaded}

\includegraphics{LRoth_A06_GLMs_files/figure-latex/unnamed-chunk-2-2.pdf}

\begin{Shaded}
\begin{Highlighting}[]
\CommentTok{#5}
\KeywordTok{bartlett.test}\NormalTok{(Ecotox}\OperatorTok{$}\NormalTok{Pub..Year }\OperatorTok{~}\StringTok{ }\NormalTok{Ecotox}\OperatorTok{$}\NormalTok{Chemical.Name)}
\end{Highlighting}
\end{Shaded}

\begin{verbatim}
## 
##  Bartlett test of homogeneity of variances
## 
## data:  Ecotox$Pub..Year by Ecotox$Chemical.Name
## Bartlett's K-squared = 139.59, df = 8, p-value < 2.2e-16
\end{verbatim}

\begin{Shaded}
\begin{Highlighting}[]
\CommentTok{#There is not equal variance among the publication years for each chemical (p<0.001).}
\end{Highlighting}
\end{Shaded}

\begin{enumerate}
\def\labelenumi{\arabic{enumi}.}
\setcounter{enumi}{5}
\tightlist
\item
  Based on your results, which test would you choose to run to answer
  your research question?
\end{enumerate}

\begin{quote}
ANSWER: Because the distribution of the data is not normal, I would run
a non-parametric test to see if the studies on the various chemicals
were conducted in different years. Since the variables are both
categorical and there are more than 2, I will use the non-paramentric
ANOVA test, Kruskal-Wallis.
\end{quote}

\begin{enumerate}
\def\labelenumi{\arabic{enumi}.}
\setcounter{enumi}{6}
\item
  Run this test below.
\item
  Generate a boxplot representing the range of publication years for
  each chemical. Adjust your graph to make it pretty.
\end{enumerate}

\begin{Shaded}
\begin{Highlighting}[]
\CommentTok{#7}
\NormalTok{Ecotox.test <-}\StringTok{ }\KeywordTok{kruskal.test}\NormalTok{(Ecotox}\OperatorTok{$}\NormalTok{Pub..Year }\OperatorTok{~}\StringTok{ }\NormalTok{Ecotox}\OperatorTok{$}\NormalTok{Chemical.Name)}
\NormalTok{Ecotox.test}
\end{Highlighting}
\end{Shaded}

\begin{verbatim}
## 
##  Kruskal-Wallis rank sum test
## 
## data:  Ecotox$Pub..Year by Ecotox$Chemical.Name
## Kruskal-Wallis chi-squared = 134.15, df = 8, p-value < 2.2e-16
\end{verbatim}

\begin{Shaded}
\begin{Highlighting}[]
\CommentTok{#There is a significant difference between the publication years of each chemical }
\CommentTok{#(ANOVA, Kruskal-Wallis; df = 8, p<0.001)}

\CommentTok{#8}
\NormalTok{Ecotox.plot <-}\StringTok{ }
\StringTok{  }\KeywordTok{ggplot}\NormalTok{(Ecotox, }\KeywordTok{aes}\NormalTok{(}\DataTypeTok{x =}\NormalTok{ Chemical.Name, }\DataTypeTok{y =}\NormalTok{ Pub..Year, }\DataTypeTok{color =}\NormalTok{ Chemical.Name)) }\OperatorTok{+}
\StringTok{  }\KeywordTok{geom_boxplot}\NormalTok{() }\OperatorTok{+}
\StringTok{  }\KeywordTok{scale_color_brewer}\NormalTok{(}\DataTypeTok{palette =} \StringTok{"Paired"}\NormalTok{) }\OperatorTok{+}
\StringTok{  }\KeywordTok{labs}\NormalTok{(}\DataTypeTok{x =} \StringTok{"Chemical Name"}\NormalTok{, }\DataTypeTok{y =} \StringTok{"Publication Year"}\NormalTok{, }\DataTypeTok{color =} \StringTok{"Chemical Name"}\NormalTok{) }\OperatorTok{+}
\StringTok{  }\KeywordTok{theme}\NormalTok{(}\DataTypeTok{axis.text.x =} \KeywordTok{element_text}\NormalTok{(}\DataTypeTok{angle =} \DecValTok{45}\NormalTok{,  }\DataTypeTok{hjust =} \DecValTok{1}\NormalTok{)) }
\KeywordTok{print}\NormalTok{(Ecotox.plot)}
\end{Highlighting}
\end{Shaded}

\includegraphics{LRoth_A06_GLMs_files/figure-latex/unnamed-chunk-3-1.pdf}

\begin{enumerate}
\def\labelenumi{\arabic{enumi}.}
\setcounter{enumi}{8}
\tightlist
\item
  How would you summarize the conclusion of your analysis? Include a
  sentence summarizing your findings and include the results of your
  test in parentheses at the end of the sentence.
\end{enumerate}

\begin{quote}
ANSWER: Studies on various neonicotinoid chemicals were conducted in
significantly different years (ANOVA, Kruskal-Wallis;X\^{}2 = 134.15, df
= 8, p\textless{}0.001).
\end{quote}

\subsection{NTL-LTER test}\label{ntl-lter-test}

Research question: What is the best set of predictors for lake
temperatures in July across the monitoring period at the North Temperate
Lakes LTER?

\begin{enumerate}
\def\labelenumi{\arabic{enumi}.}
\setcounter{enumi}{10}
\tightlist
\item
  Wrangle your NTL-LTER dataset with a pipe function so that it contains
  only the following criteria:
\end{enumerate}

\begin{itemize}
\tightlist
\item
  Only dates in July (hint: use the daynum column). No need to consider
  leap years.
\item
  Only the columns: lakename, year4, daynum, depth, temperature\_C
\item
  Only complete cases (i.e., remove NAs)
\end{itemize}

\begin{enumerate}
\def\labelenumi{\arabic{enumi}.}
\setcounter{enumi}{11}
\tightlist
\item
  Run an AIC to determine what set of explanatory variables (year4,
  daynum, depth) is best suited to predict temperature. Run a multiple
  regression on the recommended set of variables.
\end{enumerate}

\begin{Shaded}
\begin{Highlighting}[]
\CommentTok{#11}
\NormalTok{NTL.wrangled <-}
\StringTok{  }\NormalTok{NTL.chem.physics }\OperatorTok
\StringTok{  }\KeywordTok{filter}\NormalTok{(daynum }\OperatorTok{>}\StringTok{ }\DecValTok{182} \OperatorTok{&}\StringTok{ }\NormalTok{daynum }\OperatorTok{<}\StringTok{ }\DecValTok{214}\NormalTok{) }\OperatorTok
\StringTok{  }\KeywordTok{select}\NormalTok{(lakename, year4, daynum, depth, temperature_C) }\OperatorTok
\StringTok{  }\KeywordTok{na.omit}\NormalTok{()}
  
\CommentTok{#12}
\NormalTok{NTL.test <-}\StringTok{ }\KeywordTok{lm}\NormalTok{(}\DataTypeTok{data =}\NormalTok{ NTL.wrangled, temperature_C }\OperatorTok{~}\StringTok{ }\NormalTok{year4 }\OperatorTok{+}\StringTok{ }\NormalTok{daynum }\OperatorTok{+}\StringTok{ }\NormalTok{depth)}
\KeywordTok{step}\NormalTok{(NTL.test)}
\end{Highlighting}
\end{Shaded}

\begin{verbatim}
## Start:  AIC=25998.22
## temperature_C ~ year4 + daynum + depth
## 
##          Df Sum of Sq    RSS   AIC
## <none>                142056 25998
## - year4   1       201 142257 26010
## - daynum  1      1237 143293 26080
## - depth   1    402549 544605 38995
\end{verbatim}

\begin{verbatim}
## 
## Call:
## lm(formula = temperature_C ~ year4 + daynum + depth, data = NTL.wrangled)
## 
## Coefficients:
## (Intercept)        year4       daynum        depth  
##   -18.19700      0.01611      0.04024     -1.94133
\end{verbatim}

\begin{Shaded}
\begin{Highlighting}[]
\KeywordTok{summary}\NormalTok{(NTL.test)}
\end{Highlighting}
\end{Shaded}

\begin{verbatim}
## 
## Call:
## lm(formula = temperature_C ~ year4 + daynum + depth, data = NTL.wrangled)
## 
## Residuals:
##     Min      1Q  Median      3Q     Max 
## -9.6857 -3.0267  0.1055  2.9937 13.6038 
## 
## Coefficients:
##               Estimate Std. Error  t value Pr(>|t|)    
## (Intercept) -18.196998   8.741236   -2.082 0.037392 *  
## year4         0.016113   0.004353    3.701 0.000216 ***
## daynum        0.040237   0.004385    9.176  < 2e-16 ***
## depth        -1.941328   0.011728 -165.528  < 2e-16 ***
## ---
## Signif. codes:  0 '***' 0.001 '**' 0.01 '*' 0.05 '.' 0.1 ' ' 1
## 
## Residual standard error: 3.833 on 9669 degrees of freedom
## Multiple R-squared:  0.7398, Adjusted R-squared:  0.7397 
## F-statistic:  9162 on 3 and 9669 DF,  p-value: < 2.2e-16
\end{verbatim}

\begin{enumerate}
\def\labelenumi{\arabic{enumi}.}
\setcounter{enumi}{12}
\tightlist
\item
  What is the final linear equation to predict temperature from your
  multiple regression? How much of the observed variance does this model
  explain?
\end{enumerate}

\begin{quote}
ANSWER: Temperature = -5.3 + 0.009(year) + 0.04(daynum) - 1.9(depth)
This model explains 73.97\% of the variation in temperature for the
lakes in July.
\end{quote}

\begin{enumerate}
\def\labelenumi{\arabic{enumi}.}
\setcounter{enumi}{13}
\tightlist
\item
  Run an interaction effects ANCOVA to predict temperature based on
  depth and lakename from the same wrangled dataset.
\end{enumerate}

\begin{Shaded}
\begin{Highlighting}[]
\CommentTok{#14}
\NormalTok{NTL.interact <-}\StringTok{ }\KeywordTok{lm}\NormalTok{(}\DataTypeTok{data =}\NormalTok{ NTL.wrangled, temperature_C }\OperatorTok{~}\StringTok{ }\NormalTok{depth}\OperatorTok{*}\NormalTok{lakename)}
\KeywordTok{step}\NormalTok{(NTL.interact)}
\end{Highlighting}
\end{Shaded}

\begin{verbatim}
## Start:  AIC=24110.68
## temperature_C ~ depth * lakename
## 
##                  Df Sum of Sq    RSS   AIC
## <none>                        116535 24111
## - depth:lakename  8      4674 121209 24475
\end{verbatim}

\begin{verbatim}
## 
## Call:
## lm(formula = temperature_C ~ depth * lakename, data = NTL.wrangled)
## 
## Coefficients:
##                    (Intercept)                           depth  
##                        22.8748                         -2.5543  
##          lakenameCrampton Lake          lakenameEast Long Lake  
##                         2.5625                         -4.2925  
##       lakenameHummingbird Lake               lakenamePaul Lake  
##                        -2.6059                          0.7623  
##             lakenamePeter Lake            lakenameTuesday Lake  
##                         0.4321                         -2.8349  
##              lakenameWard Lake          lakenameWest Long Lake  
##                         2.4887                         -2.3347  
##    depth:lakenameCrampton Lake    depth:lakenameEast Long Lake  
##                         0.7704                          0.9181  
## depth:lakenameHummingbird Lake         depth:lakenamePaul Lake  
##                        -0.5692                          0.3698  
##       depth:lakenamePeter Lake      depth:lakenameTuesday Lake  
##                         0.5495                          0.6462  
##        depth:lakenameWard Lake    depth:lakenameWest Long Lake  
##                        -0.7207                          0.7870
\end{verbatim}

\begin{Shaded}
\begin{Highlighting}[]
\KeywordTok{summary}\NormalTok{(NTL.interact)}
\end{Highlighting}
\end{Shaded}

\begin{verbatim}
## 
## Call:
## lm(formula = temperature_C ~ depth * lakename, data = NTL.wrangled)
## 
## Residuals:
##    Min     1Q Median     3Q    Max 
## -7.683 -2.907 -0.290  2.795 16.336 
## 
## Coefficients:
##                                Estimate Std. Error t value Pr(>|t|)    
## (Intercept)                     22.8748     0.5657  40.435  < 2e-16 ***
## depth                           -2.5543     0.2330 -10.962  < 2e-16 ***
## lakenameCrampton Lake            2.5625     0.6516   3.932 8.47e-05 ***
## lakenameEast Long Lake          -4.2925     0.5992  -7.164 8.40e-13 ***
## lakenameHummingbird Lake        -2.6059     0.8262  -3.154  0.00161 ** 
## lakenamePaul Lake                0.7623     0.5787   1.317  0.18779    
## lakenamePeter Lake               0.4321     0.5773   0.749  0.45412    
## lakenameTuesday Lake            -2.8349     0.5862  -4.836 1.35e-06 ***
## lakenameWard Lake                2.4887     0.8298   2.999  0.00271 ** 
## lakenameWest Long Lake          -2.3347     0.5974  -3.908 9.36e-05 ***
## depth:lakenameCrampton Lake      0.7704     0.2379   3.239  0.00121 ** 
## depth:lakenameEast Long Lake     0.9181     0.2352   3.903 9.57e-05 ***
## depth:lakenameHummingbird Lake  -0.5692     0.2879  -1.977  0.04809 *  
## depth:lakenamePaul Lake          0.3698     0.2341   1.580  0.11417    
## depth:lakenamePeter Lake         0.5495     0.2338   2.350  0.01878 *  
## depth:lakenameTuesday Lake       0.6462     0.2345   2.755  0.00587 ** 
## depth:lakenameWard Lake         -0.7207     0.2795  -2.578  0.00995 ** 
## depth:lakenameWest Long Lake     0.7870     0.2351   3.347  0.00082 ***
## ---
## Signif. codes:  0 '***' 0.001 '**' 0.01 '*' 0.05 '.' 0.1 ' ' 1
## 
## Residual standard error: 3.474 on 9655 degrees of freedom
## Multiple R-squared:  0.7865, Adjusted R-squared:  0.7861 
## F-statistic:  2093 on 17 and 9655 DF,  p-value: < 2.2e-16
\end{verbatim}

\begin{enumerate}
\def\labelenumi{\arabic{enumi}.}
\setcounter{enumi}{14}
\tightlist
\item
  Is there an interaction between depth and lakename? How much variance
  in the temperature observations does this explain?
\end{enumerate}

\begin{quote}
ANSWER: There is a significant interaction between depth and lakename.
This model explains 78.61\% of the variation in tempearture
observations.
\end{quote}

\begin{enumerate}
\def\labelenumi{\arabic{enumi}.}
\setcounter{enumi}{15}
\tightlist
\item
  Create a graph that depicts temperature by depth, with a separate
  color for each lake. Add a geom\_smooth (method = ``lm'', se = FALSE)
  for each lake. Make your points 50 \% transparent. Adjust your y axis
  limits to go from 0 to 35 degrees. Clean up your graph to make it
  pretty.
\end{enumerate}

\begin{Shaded}
\begin{Highlighting}[]
\CommentTok{#16}
\NormalTok{NTL.plot <-}
\StringTok{  }\KeywordTok{ggplot}\NormalTok{(NTL.wrangled, }\KeywordTok{aes}\NormalTok{(}\DataTypeTok{x =}\NormalTok{ depth, }\DataTypeTok{y =}\NormalTok{ temperature_C, }\DataTypeTok{color =}\NormalTok{ lakename)) }\OperatorTok{+}
\StringTok{  }\KeywordTok{geom_point}\NormalTok{(}\DataTypeTok{alpha =} \FloatTok{0.5}\NormalTok{) }\OperatorTok{+}
\StringTok{  }\KeywordTok{geom_smooth}\NormalTok{(}\DataTypeTok{method =} \StringTok{"lm"}\NormalTok{, }\DataTypeTok{se =} \OtherTok{FALSE}\NormalTok{) }\OperatorTok{+}\StringTok{ }
\StringTok{  }\KeywordTok{ylim}\NormalTok{(}\DecValTok{0}\NormalTok{,}\DecValTok{35}\NormalTok{) }\OperatorTok{+}
\StringTok{  }\KeywordTok{labs}\NormalTok{(}\DataTypeTok{x =} \StringTok{"Depth"}\NormalTok{, }\DataTypeTok{y =} \StringTok{"Temperature"}\NormalTok{, }\DataTypeTok{color =} \StringTok{"Lake Name"}\NormalTok{)}\OperatorTok{+}
\StringTok{  }\KeywordTok{scale_color_manual}\NormalTok{(}\DataTypeTok{values =} 
    \KeywordTok{c}\NormalTok{(}\StringTok{"#80cdc1"}\NormalTok{, }\StringTok{"#7fbc41"}\NormalTok{, }\StringTok{"#de77ae"}\NormalTok{, }\StringTok{"#e08214"}\NormalTok{, }\StringTok{"#d6604d"}\NormalTok{, }\StringTok{"#4393c3"}\NormalTok{, }\StringTok{"#02818a"}\NormalTok{, }\StringTok{"#c994c7"}\NormalTok{, }\StringTok{"#e31a1c"}\NormalTok{)) }
\KeywordTok{print}\NormalTok{(NTL.plot)}
\end{Highlighting}
\end{Shaded}

\begin{verbatim}
## Warning: Removed 72 rows containing missing values (geom_smooth).
\end{verbatim}

\includegraphics{LRoth_A06_GLMs_files/figure-latex/unnamed-chunk-6-1.pdf}


\end{document}
